\documentclass[11pt,preprint, authoryear]{elsarticle}

\usepackage{lmodern}
%%%% My spacing
\usepackage{setspace}
\setstretch{1.2}
\DeclareMathSizes{12}{14}{10}{10}

% Wrap around which gives all figures included the [H] command, or places it "here". This can be tedious to code in Rmarkdown.
\usepackage{float}
\let\origfigure\figure
\let\endorigfigure\endfigure
\renewenvironment{figure}[1][2] {
    \expandafter\origfigure\expandafter[H]
} {
    \endorigfigure
}

\let\origtable\table
\let\endorigtable\endtable
\renewenvironment{table}[1][2] {
    \expandafter\origtable\expandafter[H]
} {
    \endorigtable
}


\usepackage{ifxetex,ifluatex}
\usepackage{fixltx2e} % provides \textsubscript
\ifnum 0\ifxetex 1\fi\ifluatex 1\fi=0 % if pdftex
  \usepackage[T1]{fontenc}
  \usepackage[utf8]{inputenc}
\else % if luatex or xelatex
  \ifxetex
    \usepackage{mathspec}
    \usepackage{xltxtra,xunicode}
  \else
    \usepackage{fontspec}
  \fi
  \defaultfontfeatures{Mapping=tex-text,Scale=MatchLowercase}
  \newcommand{\euro}{€}
\fi

\usepackage{amssymb, amsmath, amsthm, amsfonts}

\usepackage[round]{natbib}
\bibliographystyle{natbib}
\def\bibsection{\section*{References}} %%% Make "References" appear before bibliography
\usepackage{longtable}
\usepackage[margin=2.3cm,bottom=2cm,top=2.5cm, includefoot]{geometry}
\usepackage{fancyhdr}
\usepackage[bottom, hang, flushmargin]{footmisc}
\usepackage{graphicx}
\numberwithin{equation}{section}
\numberwithin{figure}{section}
\numberwithin{table}{section}
\setlength{\parindent}{0cm}
\setlength{\parskip}{1.3ex plus 0.5ex minus 0.3ex}
\usepackage{textcomp}
\renewcommand{\headrulewidth}{0.2pt}
\renewcommand{\footrulewidth}{0.3pt}

\usepackage{array}
\newcolumntype{x}[1]{>{\centering\arraybackslash\hspace{0pt}}p{#1}}

%%%%  Remove the "preprint submitted to" part. Don't worry about this either, it just looks better without it:
\makeatletter
\def\ps@pprintTitle{%
  \let\@oddhead\@empty
  \let\@evenhead\@empty
  \let\@oddfoot\@empty
  \let\@evenfoot\@oddfoot
}
\makeatother

 \def\tightlist{} % This allows for subbullets!

\usepackage{hyperref}
\hypersetup{breaklinks=true,
            bookmarks=true,
            colorlinks=true,
            citecolor=blue,
            urlcolor=blue,
            linkcolor=blue,
            pdfborder={0 0 0}}


% The following packages allow huxtable to work:
\usepackage{siunitx}
\usepackage{multirow}
\usepackage{hhline}
\usepackage{calc}
\usepackage{tabularx}
\usepackage{booktabs}
\usepackage{caption}
\usepackage{colortbl}

\urlstyle{same}  % don't use monospace font for urls
\setlength{\parindent}{0pt}
\setlength{\parskip}{6pt plus 2pt minus 1pt}
\setlength{\emergencystretch}{3em}  % prevent overfull lines
\setcounter{secnumdepth}{5}

%%% Use protect on footnotes to avoid problems with footnotes in titles
\let\rmarkdownfootnote\footnote%
\def\footnote{\protect\rmarkdownfootnote}
\IfFileExists{upquote.sty}{\usepackage{upquote}}{}

%%% Include extra packages specified by user
% Insert custom packages here as follows
% \usepackage{tikz}

%%% Hard setting column skips for reports - this ensures greater consistency and control over the length settings in the document.
%% page layout
%% paragraphs
\setlength{\baselineskip}{12pt plus 0pt minus 0pt}
\setlength{\parskip}{12pt plus 0pt minus 0pt}
\setlength{\parindent}{0pt plus 0pt minus 0pt}
%% floats
\setlength{\floatsep}{12pt plus 0 pt minus 0pt}
\setlength{\textfloatsep}{20pt plus 0pt minus 0pt}
\setlength{\intextsep}{14pt plus 0pt minus 0pt}
\setlength{\dbltextfloatsep}{20pt plus 0pt minus 0pt}
\setlength{\dblfloatsep}{14pt plus 0pt minus 0pt}
%% maths
\setlength{\abovedisplayskip}{12pt plus 0pt minus 0pt}
\setlength{\belowdisplayskip}{12pt plus 0pt minus 0pt}
%% lists
\setlength{\topsep}{10pt plus 0pt minus 0pt}
\setlength{\partopsep}{3pt plus 0pt minus 0pt}
\setlength{\itemsep}{5pt plus 0pt minus 0pt}
\setlength{\labelsep}{8mm plus 0mm minus 0mm}
\setlength{\parsep}{\the\parskip}
\setlength{\listparindent}{\the\parindent}
%% verbatim
\setlength{\fboxsep}{5pt plus 0pt minus 0pt}



\begin{document}

\begin{frontmatter}  %

\title{Theory of Statistics Likelihood Assigment}

\author[Add1]{Sean Soutar STRSEA001}
\ead{sean.soutar@gmail.com}

\author[Add2]{Fabio Fehr FHRFAB001}
\ead{FHRFAB001@myuct.ac.za}




\address[Add1]{UCT Statistics Honours, Cape Town, South Africa}
\address[Add2]{UCT Statistics Honours, Cape Town, South Africa}


\begin{abstract}
\small{
This project will explore the Accidents dataset and try fit a Poisson,
Negative Binomial, Mixture of 2 Poissons and zero inflated Poisson
models to the data. The model with the strongest support will be chosen
and discussed. Profile likelihoods and confidence intervals for the
parameters will be found and displayed of the chosen model.
}
\end{abstract}

\vspace{1cm}

\begin{keyword}
\footnotesize{
Likelihood \sep Overdispersion \sep Soek \\ \vspace{0.3cm}
\textit{JEL classification} 
}
\end{keyword}
\vspace{0.5cm}
\end{frontmatter}



%________________________
% Header and Footers
%%%%%%%%%%%%%%%%%%%%%%%%%%%%%%%%%
\pagestyle{fancy}
\chead{}
\rhead{}
\lfoot{}
\rfoot{\footnotesize Page \thepage\\}
\lhead{}
%\rfoot{\footnotesize Page \thepage\ } % "e.g. Page 2"
\cfoot{}

%\setlength\headheight{30pt}
%%%%%%%%%%%%%%%%%%%%%%%%%%%%%%%%%
%________________________

\headsep 35pt % So that header does not go over title




\section{Introduction}\label{introduction}

This assignment is an explorative report on a dataset containing
accident counts. The aim of the report is to find and fit a model which
accurately describes the accident dataset. This report will first
explore the data then fit different adequate distributions and choose
the most appropriate one. Once a model has been selected the profile
likelihood and confidence intervals will be programmed and calculated
from from first principles. The results will then be analysed critically
and conclusions will be made and consider further considerations in the
study.

\subsection{Exploratory data analysis}\label{exploratory-data-analysis}

To better understand our data this report shall explore the following
properties; Firstly we examine the type of data within the accidents
dataset and discuss whether our data is discrete ordinal or continuous.
After the symmetry of the data and bounds will be discussed. This leads
the exploration to outliers and extreme values.

\includegraphics{likelihood_files/figure-latex/unnamed-chunk-1-1.pdf}

\subsubsection{Data type}\label{data-type}

There are many instances where zero accidents were observed. This
accounts for approximately 25.18\% of the data. This suggests that the
zero-inflated Poisson should be considered as this proportion is much
higher than what would be expected of a regular Poisson distribution.
The accident counts are discrete random variables. Specifically, they
are discrete positive definite random variables on the interval
\(R \in \{0;+ \infty\}\). Summary statistics of the data are shown
below.

\begin{longtable}[]{@{}rrr@{}}
\toprule
Mean & Variance & Median\tabularnewline
\midrule
\endhead
6.917892 & 85.08584 & 4\tabularnewline
\bottomrule
\end{longtable}

In the Poisson distribution, the mean should equal the variance. The
sample variance far exceeds the sample mean. This indicates
overdispersion if the Poisson distribution were to be used. This is when
the observations are more variable than what would be expected. This
suggests that alternative count models and mixture distributions should
be used.

\subsubsection{Symmetry}\label{symmetry}

This property is visually seen in the histogram and boxplot. All counts
are greater than zero with a median value of 4 accidents. The largest
accident observed is 70 accidents. THe histogram shows that the data are
non-symetrical and positively skewed which is usually expected of count
data.

\subsubsection{Outliers}\label{outliers}

From the boxplot it clear that many outliers exist. One common method of
classifying a point as an extreme value or outlier is if it falls more
than 1.5 times the inner-quartile range above the upper quartile. The
proportion of outliers within our data set amount to 15.26\%.

\section{Methods}\label{methods}

\subsection{Model Formulation}\label{model-formulation}

The data is discrete, asymmetric, positive definite, contains many
positive outliers and many zeros. This would suggest distributions such
as Poisson, Negative Binomial, mixture distribution of 2 Poissons and a
zero inflated Poisson.

\subsection{Akiake Information Coefficient
(AIC)}\label{akiake-information-coefficient-aic}

should we look at biC? The AIC metric can be used to compare models from
different families of distributions. They can be used to compare
relative goodness of fit between models. A lower AIC value indicates a
better fitting model.

\(\text{AIC} = -2l(\hat{\theta}) + 2\text{p} \\ p = \text{Number of estimated parameters}\)

\subsubsection{Poisson}\label{poisson}

\begin{align*} 
p(x) & =  \frac{e^{-\lambda} \lambda^x}{x!},\ \ x\in \{0,1,\ldots,\infty\},\lambda>0 \\
\\
L(\lambda|x) & = \prod_{i=1}^n p(x_i) \\
\\
L(\lambda|x) & =\dfrac{e^{-n\lambda}\lambda^{\sum_{i=1}^n x_i}}{\prod_{i=1}^n x_i!}\\
\\
l(\lambda|x) & =-n\lambda +  \left(\sum_{i=1}^n x_i\right)\ln \lambda - \sum_{i=1}^{n}\ln(x_i!)
\end{align*}

The Poisson is characterised by the \(\lambda\) parameter which denotes
the population average rate of event occurence. In this context it would
be the average number of accidents per unit time frame.

\begin{longtable}[]{@{}rr@{}}
\toprule
Lambda Estimate & AIC\tabularnewline
\midrule
\endhead
6.917892 & 20263.64\tabularnewline
\bottomrule
\end{longtable}

\subsubsection{Negative Binomial}\label{negative-binomial}

\begin{align*} 
p(x) & =  {\frac {\Gamma (r+x)}{x!\,\Gamma (r)}}\left({\frac {m}{r+m}}\right)^{x}\left({\frac {r}{r+m}}\right)^{r}\quad {\text{for }}x=0,1,2,\dotsc \\
\\
L(m,r|x) & = \prod_{i=1}^n p(x_i) \\
\\
L(m,r|x) & ={[\frac{1}{\Gamma (r)}]}^{n} \prod_{i=1}^{n}{\frac{\Gamma (r+x_i)}{x_{i}!}} (\frac{m}{r + m})^{\Sigma_{i=1}^n x_i} (\frac{r}{r + m})^{nr}   \\
\\
l(m,r|x) & = -n\ln[\Gamma (r)] + \sum^{n}_{i=1} \ln(\Gamma (r + x_i)) -\sum^{n}_{i=1}\ln x_i! + \sum^{n}_{i=1} x_{i} \ln (\frac{m}{r + m}) + nr \ln (\frac{r}{r + m})
\end{align*}

This parameterisation of the negative binomial is characterised by the
mean parameter m and the shape parameter r.

\begin{longtable}[]{@{}rrr@{}}
\toprule
M Mean Estimate & r Shape Estimate & AIC\tabularnewline
\midrule
\endhead
6.810803 & 0.592616 & -9620.917\tabularnewline
\bottomrule
\end{longtable}

-define all parameters -fit to the data

\subsubsection{Mixture of 2 poissons}\label{mixture-of-2-poissons}

-Likelihood -define all parameters -loglikelihood

-fit to the data

-Here I am assuming the mixture will be poisson with rate = sample mean
and the other poisson will have a rate of 0.1 to take into account the
zero inflation ?

\subsubsection{Zero inflated Poisson}\label{zero-inflated-poisson}

-Likelihood -loglikelihood -define all parameters -fit to the data -We
can use optimisers but we must program the likelihoods ourselves

\subsection{Model Selection}\label{model-selection}

-compare models and choose the best one -Illustrate how good the model
is

-We need to reparameterize parameters so that they are unbounded

\subsection{Profile Likelihood \& Confidence
Intervals}\label{profile-likelihood-confidence-intervals}

-Plot likelihood surface (two parameters at a time if necessary, fixing
the other parameters at their MLEs).

-Must be program the profile likelihoods, CI's ourselves

\section{Results}\label{results}

\section{Conclusion}\label{conclusion}

-What are the next steps and how can we improve the models

\section{References}\label{references}

% Force include bibliography in my chosen format:
\newpage
\nocite{*}
\bibliography{}





\end{document}
